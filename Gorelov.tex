\documentclass{article}
\usepackage{graphicx} % Required for inserting images

%\usepackage[russian, english]{babel}
\usepackage[english,russian]{babel}
\usepackage{amsmath}
\usepackage{amssymb}
\usepackage{MnSymbol}
%\usepackage{latexsim}
\usepackage[active]{srcltx}
\usepackage[final]{pdfpages}
\usepackage{graphicx}
\usepackage{wrapfig}
\usepackage{amsmath}
\usepackage{MnSymbol}
\usepackage{wasysym}
\usepackage{amsmath}
\usepackage{tikz}
\usepackage{subfig}
\usepackage{listings}
\usepackage{bigints}
\usepackage{changepage}
\usepackage[utf8]{inputenc}
\usepackage[colorlinks=false,unicode=true]{hyperref}
\usepackage[labelsep=period]{caption}
\usetikzlibrary{arrows,decorations.pathmorphing,backgrounds,positioning,fit,petri}
\usepgflibrary{arrows} % LATEX and plain TEX and pure pgf
\usetikzlibrary{arrows} % LATEX and plain TEX when using Tik Z
\definecolor{whitesmoke}{rgb}{0.9,0.9,0.9} \definecolor{whitesmoke2}{rgb}{0.8,0.8,0.8}

\def\e{\varepsilon}
\newcommand{\pp}{\varphi}
\catcode`@=11
\def\caseswithdelim#1#2{\left#1\,\vcenter{\normalbaselines\m@th
  \ialign{\strut$##\hfil$&\quad##\hfil\crcr#2\crcr}}\right.}% you might like it without the \strut
\catcode`@=12
%
\def\bcases#1{\caseswithdelim[{#1}}
\def\vcases#1{\caseswithdelim|{#1}}
%
% Пакеты по желанию (самые распространенные)
% Хитрые мат. символы
\usepackage{euscript}
% Таблицы
\usepackage{longtable}
\usepackage{makecell}

\title{Исследование одной двумерной системы автономных дифференциальных уравнений с малым параметром возмущения}
\author{Василий Горелов}

\begin{document}

\maketitle

\section{Вступление}
Имеется первоначальная система без возмущений:
\begin{equation}
\begin{cases}
      \dot x=-xy^2+x+y,\\
      \dot y=-x-y+x^2y.
    \end{cases}
    \label{eq:first}
\end{equation}

После добавления определенного малого возмущения 3-его порядка изначальная система (\ref{eq:first}) примет вид:
\begin{equation}
\begin{cases}
      \dot x=-xy^2+x+y-\epsilon x^3,\\
      \dot y=-x-y+x^2y+\epsilon y^3.
    \end{cases}
    \label{eq:second}
\end{equation}

Поскольку система (\ref{eq:first}) была достаточно изучена в предыдущей работе, то сосредоточимся на возмущенной системе (\ref{eq:second}).\\

Для начала составим интегрируемую комбинацию. Используя те же соображения, что и в работе \cite{basov}, получаем два соотношения:
\begin{equation*}
\dot{xy} = (x^2-y^2)(xy(1-\epsilon)-1)\hbox{,\hspace{10pt}} \dot x^2+\dot y^2 = 2(x^2-y^2)(1-\epsilon(x^2+y^2)).
\end{equation*}

Подставляя отсюда $x^2-y^2$ из второго уравнения в первое получаем интегрируемую комбинацию $d(x^2+y^2)2^{-1}(1-\epsilon(x^2+y^2))^{-1}=d(xy)(xy(1-\epsilon)-1)^{-1}$ ,или $xy(1-\epsilon)=1$, или $\epsilon(x^2+y^2)=1$.\\

Интегрируя, получаем первый интеграл системы (\ref{eq:second}) 

\begin{equation}
    (1-\epsilon)^{-1}\ln|(1-\epsilon)xy -1|+2^{-1}\epsilon^{-1}\ln{|\epsilon(x^2+y^2)-1|} = C
\end{equation}

Минуя все вычисления, имеем семь особых точек системы (\ref{eq:second}) при $\epsilon > 0$: $(0,0)$, $(\pm\sqrt{2(1+\epsilon)^{-1}}$, $\pm\sqrt{2(1+\epsilon)^{-1}})$, ()

 \begin{thebibliography}{1}
\bibitem{basov} Басов В.В. \flqq 	Обыкновенные дифференциальные уравнения. Лекции и практические занятия\frqq. 2023, pp. 1223-1235.
\end{thebibliography}
\end{document}